\documentclass{msi-exam}

%----------------------------------
% Comment/uncomment the following to show answers / grading comments.
%----------------------------------
\setboolean{showanswer}{true}
\setboolean{showgrade}{true}

%----------------------------------
% Setup Exam Title page 
%----------------------------------

\newcommand{\examTitleOne}   {Semester 2 --- Mid-Semester, 2016} 
\newcommand{\examTitleTwo}   {MATH1013 --- Advanced Mathematics and Applications 1}
\newcommand{\examTitleThree} {Book A --- Calculus}
\newcommand{\examFooter}     {MATH1013 --- Book A}
\newcommand{\examTime}{120}
\newcommand{\examRead}{15}
\newcommand{\examConditions}{%
  \begin{itemize}
    \item Central examination.
    \item Students must return the examination paper at the end of the examination.
    \item This examination paper is not available to the ANU Library archives.
  \end{itemize}
}
\newcommand{\examPermitted}{% 
  \begin{itemize}
    \item One A4 page with hand written notes on both sides.\\ (This A4 page is to cover both Algebra and Calculus.)
    \item Unmarked English-to-foreign-language dictionary (no approval from MSI required).
    \item {\bf No electronic aids are permitted e.g. laptops, phones, calculators.}
  \end{itemize}
}
\newcommand{\examMaterial}{%
  \begin{itemize}
    \item Scribble Paper.
  %\item Add extra items here as required: e.g., Trigonometric Cheat Sheet.
  \end{itemize} 
}
\newcommand{\examImportantNotes}{%
\begin{itemize}   
  \item Answer the Calculus questions in Book A, and the Algebra questions in Book B, in the spaces provided.
  \item The Algebra and Calculus sections are worth a total of 50 points each, with the value of each question as shown. It is recommended that you spend equal time on the Calculus and the Algebra papers.
  \item A good strategy is not to spend too much time on any question.
  Read them through first and attack them in the order that allows you
  to make the most progress.
  \item \textbf{\it You must justify your answers. Please be neat.}
\end{itemize}
}


%================================================================================
\begin{document}
%================================================================================

%% The question environment takes an argument, specifying the total points for that question.
%% This automatically generates grading boxes on the title page.
\begin{question}{10}
\begin{enumerate}
\item Using the definition of $\sinh x$ in terms of exponentials, find the indefinite integral
\begin{equation*}
\int\sinh x\,dx\,.\tag*{\subpoints{2}}
\end{equation*}

\workspace 


\item Evaluate $\displaystyle{\int_{1}^{\sqrt3}\frac{1}{(1+x^2)(\tan^{-1} x)^2}\,dx\,.}$
\subpoints{4}

\workspace 

\begin{answer}
Make the substitution $u=\tan^{-1}x$. Then $du=\frac{1}{1+x^2}\, dx$, so 
\begin{align*}
\int_{1}^{\sqrt3}\frac{1}{(1+x^2)(\tan^{-1} x)^2}\,dx
  &= \int_{\pi/4}^{\pi/3} \frac{du}{u^{2}} \\[5pt]
  &= \left[-u^{-1}\right]_{\pi/4}^{\pi/3} \\[5pt]
  &= -\frac3\pi+\frac4\pi\\[5pt]
  &=\frac1\pi\,. 
\end{align*}
\end{answer}

\begin{grading}
\item 1 point for trying a substitution
\item 1 point for correctly performing the change of variables
\item 1 point for computing the resulting integral
\item 1 point for the right answer!
\end{grading}

\newpage
\item Evaluate $\displaystyle{\int_1^{2} (\ln x)^2\,dx\,.}$
\subpoints{4}

\workspace 

\begin{answer}
 
Let $U_1=(\ln x)^2$ and $dV_1 = dx$. Then $dU_1 = \frac{2}{x}\ln x\,dx$ and $V_1 = x$, so integration by parts gives
\begin{align*}
\int_1^{2}(\ln x)^2\,dx\,&=\,\bigg[x(\ln x)^2\bigg]_1^2 - \int_1^2 x\tfrac{2}{x}\ln x\,dx\,=\,2(\ln 2)^2 - 2\int_1^2\ln x\,dx\,.
\end{align*}
Now let $U_2=\ln x$ and $dV_2 = dx$. Then $dU_2 = \frac1x\,dx$ and $V_2 = x$, so integration by parts gives
\begin{align*}
\int_1^2\ln x\,dx\,&=\,\bigg[x\ln x\bigg]_1^2 - \int_1^2 x\tfrac1x\,dx\\
&=\,2\ln 2 - \int_1^2 dx\\
&=\,2\ln 2-1\,.
\end{align*}
Hence
\begin{align*}
\int_1^2(\ln x)^2\,dx\,&=\,2(\ln 2)^2 - 4\ln 2+2\,.
\end{align*}

\end{answer}

\end{enumerate}
\end{question}

\begin{question}{10}
Suppose $\alpha>1$, $K\ge 0$, and that $f:\R\to\R$ has the property that, for all $x,y\in\R$,
\[
|f(x)-f(y)|\,\le\, K|x-y|^{\alpha}\,.
\]
Prove that $f$ is constant.\qrule
\emph{
  Hints: for this question you may use the facts that
  \begin{itemize}
    \item for all $r>0$, the mapping $t\mapsto t^r$ is continuous on $[0,\infty)$, and
    \item if $g:\R\!\setminus\!\{c\}\to\R$, then $\lim\limits_{x\to c}g(x)=0$ if and only if $\lim\limits_{x\to c}|g(x)|=0$.
  \end{itemize}
  Given $c\in\R$, consider $h:\R\!\setminus\!\{c\}\to\R$ defined by $h(x)=\frac{f(x)-f(c)}{x-c}$.\\[0.5em]
  Finally, please note that $\alpha>1$, NOT $\alpha>0$.
}

\workspace

\extraspace

\end{question}


\end{document}
